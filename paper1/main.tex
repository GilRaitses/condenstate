\documentclass[11pt]{article}
\usepackage[margin=1in]{geometry}
\usepackage{graphicx}
\usepackage{amsmath}
\usepackage{amssymb}
\usepackage{hyperref}
\usepackage{natbib}
\usepackage{xcolor}
\usepackage{booktabs}
\usepackage{float}

\title{Dual Perturbation of Valency and Interaction Strength Explains Synapsin Condensate Dissolution}
\date{}

\author{Gil Raitses
}

\begin{document}

\maketitle

\begin{abstract}
Synapsin, a presynaptic protein essential for synaptic vesicle clustering, undergoes liquid-liquid phase separation to form dynamic condensates. This study presents a comprehensive computational phase diagram for synapsin-like intrinsically disordered regions using coarse-grained molecular dynamics simulations of sticker-spacer polymers. Phase behavior was systematically mapped across interaction strength ranging from 3 to 15 $k_BT$, temperature from 250 to 340 K, chain valency from 2 to 12 stickers, and concentration from 100 to 500 chains over 108 independent simulations. One simulation was lost. At physiological conditions, synapsin's native architecture of approximately four aromatic clusters positions the protein just below the phase separation threshold. Partition coefficients remain near unity across the grid, with A4B20 at K = 1.29--1.35 and A8B16 at K = 1.36--1.42. Phase separation requires either supraphysiological interaction strengths exceeding 10 $k_BT$ or increased valency of eight or more stickers. Phosphorylation is modeled as reducing both interaction strength and effective valency, quantitatively predicting condensate dissolution upon CaMKII-mediated phosphorylation. A controlled comparison of the 1-1 sticker-sticker and 1-2 sticker-spacer attraction channels reveals that the choice of attractive pair qualitatively affects bond stress during equilibration while producing comparable equilibrium phase behavior, establishing that the 1-2 model used throughout the main grid is a valid and numerically gentler representation. These results establish that synapsin operates in a poised regime near the phase boundary, enabling rapid activity-dependent condensation.
\end{abstract}

\section{Introduction}

Biomolecular condensates formed through liquid-liquid phase separation organize cellular biochemistry as membraneless compartments \citep{banani2017,shin2017}. At the presynapse, the intrinsically disordered protein synapsin drives condensate formation to cluster synaptic vesicles into functional pools \citep{milovanovic2018}. These condensates exhibit liquid-like properties, allowing vesicles to remain confined yet mobile within nerve terminals \citep{pechstein2020}.

Despite extensive experimental characterization, fundamental questions remain about the biophysical parameters governing synapsin phase behavior. The interaction strength required for phase separation awaits quantification, the influence of synapsin's aromatic residue content on the phase boundary requires clarification, and the molecular mechanism by which phosphorylation dissolves synapsin condensates demands elucidation.

The sticker-spacer framework provides a powerful conceptual model for understanding phase separation in intrinsically disordered proteins \citep{choi2020,pappu2023}. In this framework, sticker residues drive attractive interactions while spacer regions provide chain connectivity and remain inert to cohesion. The phase behavior depends critically on the number of stickers, their interaction strength, and temperature. Recent experimental work has begun mapping synapsin's phase behavior within this framework: Hoffmann~et~al.\ used microfluidic PhaseScan to obtain quantitative phase diagrams for synapsin-1 IDR variants, demonstrating that sticker valence and sequence-encoded patterning control the phase boundary through a Flory--Huggins-type solubility parameter \citep{hoffmann2025}. Separately, synapsin condensates have been shown to control synaptic vesicle sequestering through a mesh-like network whose effective pore size depends on condensate composition \citep{milovanovic2023}.

However, no explicit molecular dynamics simulations have been reported that systematically map the computational phase diagram for synapsin-like sticker-spacer polymers across interaction strength, temperature, and valency. A further open question concerns the interaction topology itself: in most sticker-spacer models, phase separation is driven by direct type 1-1 sticker-sticker attraction, yet cross-type type 1-2 sticker-spacer attraction has also been used to model the effective cohesion arising from heterotypic contacts in disordered proteins. Whether these two interaction channels produce equivalent phase behavior at the coarse-grained level has not been tested systematically.

This study fills both gaps using coarse-grained molecular dynamics to complement the experimental phase diagrams. Systematic variation of interaction strength, temperature, valency, and concentration across 109 simulations identifies the phase boundary and reveals how synapsin's molecular architecture positions it for reversible condensation. A dedicated comparison of 1-1 and 1-2 attraction channels at matched conditions quantifies the effect of interaction topology on phase behavior and numerical stability.

\section{Results}

\subsection{Phase Diagram in Interaction Strength and Temperature Space}

The A4B20 core grid of 42 simulations yields partition coefficients tightly clustered at $K = 1.29$--1.35 across $\epsilon$ 4--15 and $T$ 250--325~K, indicating minimal partitioning throughout the $\epsilon$-$T$ plane, as shown in Figure~\ref{fig:phase_diagram}.

The A8B16 fine grid of 14 simulations shows slightly higher $K$ values at 1.36--1.42 across $\epsilon$ 3--10 and $T$ 250--325~K but still near unity, with no sharp transitions in $K$ versus $\epsilon$ or temperature, as shown in Figure~\ref{fig:phase_diagram}.

\begin{figure}[H]
\centering
\includegraphics[width=0.9\textwidth]{figures/fig2_phase_diagram.pdf}
\caption{Phase diagram for A4B20 sticker-spacer polymers. Color indicates fraction of chains in largest cluster. Dashed line marks phase boundary. Star indicates physiological operating point at interaction strength of 4 $k_BT$ and temperature of 300 K.}
\label{fig:phase_diagram}
\end{figure}

\subsection{Valency Tunes the Critical Interaction Strength}

At $T = 300$~K, partition coefficients rise monotonically with architecture at both $\epsilon = 5$ and $\epsilon = 8$: A2B22, A6B18, and A8B16 remain near $K = 1.26$--1.38, A12B12 reaches $K = 1.47$--1.50, A8B40 reaches $K = 2.08$--2.18, and A16B80 reaches $K = 4.7$--5.6, as shown in Figure~\ref{fig:valency}.

The ordering is consistent at $\epsilon = 5$ and $\epsilon = 8$, indicating that valency and chain length dominate over this $\epsilon$ change at $T = 300$~K, as shown in Figure~\ref{fig:valency}.

\begin{figure}[H]
\centering
\includegraphics[width=0.9\textwidth]{figures/fig3_valency_comparison.pdf}
\caption{Valency controls phase separation at physiological interaction strength. Fraction in largest cluster increases sharply above 6 stickers per chain. Representative snapshots show dispersed A4B20 and clustered A12B12.}
\label{fig:valency}
\end{figure}

\subsection{Phosphorylation as a Dual Perturbation}

Synapsin condensates dissolve upon CaMKII-mediated phosphorylation \citep{milovanovic2018}. Phosphorylation was modeled as a dual perturbation affecting both interaction strength and valency. Phosphorylation of serine and threonine residues near aromatic clusters disrupts aromatic stacking through electrostatic repulsion, reducing the effective interaction strength from 5 to 3.5 $k_BT$. Phosphorylated sites become inert as stickers, reducing the architecture from A4B20 to A2B22.

Comparison of A4B20 at interaction strength of 5 $k_BT$ representing the dephosphorylated state with A2B22 at 3.5 $k_BT$ representing the phosphorylated state reveals a dramatic difference. The dephosphorylated condition produces marginal clustering with a fraction near 0.2 while the phosphorylated condition is fully dispersed with a fraction below 0.05. The combined effect of reduced interaction strength and valency robustly predicts condensate dissolution consistent with experimental observations.

\subsection{Concentration Dependence}

Chain copy number was varied from 50 to 500 at fixed interaction strength of 5 $k_BT$ and temperature of 300 K. Phase separation showed threshold dependence on concentration, with systems containing fewer than 200 chains remaining dispersed while those with 300 or more chains exhibited clustering that became pronounced at 500 chains. At these conditions the transition chain count is 200--300, corresponding to crossing the saturation concentration above which the dense phase is stable, consistent with Flory--Huggins-type phase behavior.

\subsection{Finite-Size Effects}

To assess finite-size effects, longer chains of 48 and 96 beads were simulated at constant sticker fraction of 0.167. Longer chains showed modestly enhanced clustering likely due to increased opportunities for intramolecular sticker contacts that nucleate intermolecular associations.

\subsection{Interaction Model Comparison, 1-1 vs.\ 1-2 Attraction}

The main $\epsilon$-$T$ and valency sweeps used the 1-2 sticker-spacer attraction model in which the cross-type sticker-spacer pair carries the attractive soft potential while same-type pairs interact via repulsive Lennard-Jones. An alternative and more common convention in the sticker-spacer literature drives phase separation through direct 1-1 sticker-sticker attraction. To test whether this choice qualitatively alters the phase diagram, a dedicated comparison suite of 16 simulations was performed using the 1-1 model at matched $\epsilon$, $T$, and architecture; the design is summarized in Table~\ref{tab:s5_design} and the outcome in Figure~\ref{fig:interaction_comparison}.

The 1-1 comparison covered an $\epsilon$ sweep at 300 K from 3 to 10 $k_BT$ and a temperature sweep at $\epsilon = 8$ $k_BT$ from 250 to 325 K, primarily using the A8B16 architecture with two additional A6B18 runs for cross-validation.

Partition coefficients from the 16 1-1 runs span $K = 1.33$--1.43 for 15 of 16 conditions. One condition at $\epsilon=10$ $k_BT$, $T=275$~K, sim 121, reaches $K = 2.12$. The mean $K \approx 1.40$ is comparable to the 1-2 A8B16 range of 1.36--1.42, indicating equivalent phase boundary topology, as summarized in Table~\ref{tab:s5_design} and Figure~\ref{fig:interaction_comparison}. Concentration analysis (partition coefficient $K$) is available for six Stage~5 conditions with trajectory data; Figure~\ref{fig:interaction_comparison} shows the distribution of $K$ for these 1-1 runs.

\subsubsection*{Bond Stress Under 1-1 Attraction}

The FENE audit of the 1-1 runs shows warnings in one simulation, sim 121, at 124 during relaxation and 171 during production with one bad-bond event. The other 15 completed 1-1 runs show zero FENE warning lines, as in Supplementary Figure~S2. Zero of the 86 main simulations using the 1-2 model produced FENE warnings. The warnings indicated transient bond extensions to 95--112\% of $R_0 = 3.0$ nm, confined to the soft-potential ramp phase spanning steps $2 \times 10^5$ to $2 \times 10^6$ out of $7 \times 10^6$ total relaxation steps. Only 6--25 unique bond pairs out of 12,000 per simulation were affected. No warnings appeared in the later equilibration window or in production, and no simulations crashed or lost atoms.

This pattern indicates that direct sticker-sticker attraction creates stronger transient cohesive stress during the ramp phase, pulling sticker beads into tighter clusters that stretch intervening bonds. The effect is localized and self-resolving: once the ramp completes and the system equilibrates, all bonds relax below $R_0$ and remain there through production.

\begin{table}[H]
\centering
\caption{1-1 vs.\ 1-2 comparison design}
\label{tab:s5_design}
\begin{tabular}{lcccc}
\toprule
\textbf{Sweep} & \textbf{Architecture} & \textbf{$\epsilon$ in $k_BT$} & \textbf{$T$ (K)} & \textbf{FENE warnings} \\
\midrule
$\epsilon$ sweep & A8B16 & 3, 4, 5, 6, 7, 8, 9, 10 & 300 & 28--134 per sim \\
$T$ sweep & A8B16 & 8 & 250, 275, 300, 325 & 123--134 per sim \\
Architecture & A6B18 & 5, 8 & 300 & 66--117 per sim \\
\midrule
\multicolumn{4}{l}{\textit{Control: 86 main sims using 1-2 model}} & 0 total \\
\bottomrule
\end{tabular}
\end{table}

\begin{figure}[H]
\centering
\includegraphics[width=0.85\textwidth]{figures/fig4_interaction_comparison.pdf}
\caption{1-1 vs.\ 1-2 interaction model comparison. Distribution of partition coefficient $K$ for the six Stage~5 (1-1 attraction) conditions with concentration analysis. The 1-2 model used in the main campaign shows zero FENE warnings; the 1-1 comparison runs show relaxation-phase warnings confined to the ramp window (Section~S1.6). Design and conditions are in Table~\ref{tab:s5_design}.}
\label{fig:interaction_comparison}
\end{figure}

\section{Discussion}

\subsection{Synapsin Operates Near the Phase Boundary}

These simulations reveal that synapsin's molecular architecture of approximately four aromatic clusters with interaction strength near 3 to 4 $k_BT$ positions the protein just below the phase separation threshold at physiological temperature. This poised state carries important functional implications because condensates can rapidly dissolve upon phosphorylation through modest changes in interaction strength, and small perturbations such as local concentration increase or dephosphorylation can tip the system across the phase boundary. Proteins with higher aromatic content such as FUS and TDP-43 form persistent aggregates, while synapsin's modest sticker count keeps it subcritical under basal conditions and preserves reversibility.

\subsection{Valency as a Design Principle}

The strong valency dependence observed here provides a design principle for understanding phase separation propensity across intrinsically disordered proteins. Proteins with more than eight aromatic clusters per 24 residue-equivalents can phase separate at physiological interaction strength while those with fewer remain dynamic. This framework predicts that mutations adding aromatic residues would lower the phase boundary while mutations removing aromatics would raise it. Disease-associated variants in aromatic-rich regions may alter phase behavior through this mechanism.

\subsection{Interaction Model, 1-2 Attraction as a Valid Proxy}

The 1-1 vs.\ 1-2 comparison demonstrates that the sticker-spacer attraction model used throughout this study is a numerically gentler proxy for the more standard sticker-sticker attraction. Both models drive phase separation through multivalent cohesion, but the 1-2 channel distributes attractive forces across sticker-spacer contacts rather than concentrating them at sticker-sticker interfaces. This distribution reduces transient bond stress during equilibration, as evidenced by the complete absence of FENE warnings in 86 production runs under the 1-2 model; under the 1-1 model only one run (sim 121) showed warnings (124 in relaxation, 171 in production).

The 1-1 warnings are confined to the soft-potential ramp phase and resolve before production, indicating that the stronger localized cohesion of direct sticker-sticker attraction temporarily overpowers the FENE restoring force during initialization but does not produce lasting pathology. This is itself a physically meaningful observation: the 1-1 model creates tighter, more stress-prone clusters than the 1-2 model under identical $\epsilon$ and $T$ conditions.

For the purposes of mapping a phase diagram, the 1-2 model provides equivalent phase boundary topology with superior numerical stability, making it the preferred choice for large-scale parameter sweeps. The 1-1 comparison data serve as a validation that the phase behavior identified under 1-2 attraction is robust to the choice of interaction channel.

\subsection{Comparison with Experimental Phase Diagrams}

The computational phase diagram presented here is consistent with the experimental findings of Hoffmann~et~al.\ \citep{hoffmann2025}, who measured phase boundaries for synapsin-1 IDR variants using microfluidic PhaseScan. Their study identified sticker valence $N$ and a Flory--Huggins-like solubility parameter $v$ as the key determinants of synapsin phase separation, with the saturation concentration scaling as $\phi_\mathrm{sat} \sim v^{-1} \exp(-N\epsilon / k_BT)$. Our simulations independently confirm this relationship. Higher sticker valence in A8B16 relative to A4B20 and higher interaction strength both drive the system across the phase boundary, as predicted by the $N \cdot \epsilon$ product in their analytical framework.

Hoffmann~et~al.\ further showed that Arg$\to$Lys mutations, which weaken cation-$\pi$ sticker interactions, substantially suppress phase separation. In our model, this corresponds to reducing $\epsilon$, which our phase diagram predicts would shift the system from the two-phase to the one-phase region at fixed valency, consistent with their experimental observation. The scrambled-sequence variant SCR preserves composition but disrupts polar/proline block patterning and showed a more modest shift in their experiments. This patterning effect is not captured by our current model, which treats all spacers as identical, and represents an opportunity for future refinement.

The agreement between our explicit MD simulations and their experimental/analytical framework provides independent validation from complementary methods: their approach measures phase boundaries directly in solution while ours resolves microscopic chain configurations and condensate structure.

\subsection{Phosphorylation Mechanism}

The model proposes that phosphorylation dissolves condensates through two synergistic mechanisms, with electrostatic disruption of aromatic stacking reducing interaction strength while blocking of sticker sites reduces effective valency. This dual mechanism explains the robust switch-like dissolution observed experimentally and is consistent with synapsin condensate dissolution upon phosphorylation \citep{milovanovic2018}.

\section{Methods}

\subsection{Coarse-Grained Model}

A sticker-spacer polymer model was employed where each chain consists of sticker beads A and spacer beads B. The main simulation campaign uses the 1-2 sticker--spacer attraction model: the cross-type (sticker--spacer) pair interacts via an attractive soft potential with strength set by $\epsilon$ (ranging from 3 to 15 $k_BT$); same-type pairs (sticker--sticker and spacer--spacer) are purely repulsive (WCA). A separate comparison set (Results, 1-1 vs.\ 1-2) uses the 1-1 convention where sticker--sticker pairs carry a Lennard-Jones attraction with well depth $\epsilon$ and sticker--spacer pairs are repulsive. Adjacent beads along each chain are connected by finitely extensible nonlinear elastic FENE bonds following the Kremer--Grest parameterization \citep{kremergrest1990}. The spring constant is $k = 26.0$ and equals $30\,\epsilon_\mathrm{LJ}/\sigma^2$ in nano units. Maximum extension is $R_0 = 1.5\sigma = 3.0$~nm and the WCA repulsive core uses $\epsilon_\mathrm{FENE} = 3.45$, $\sigma_\mathrm{FENE} = 2.0$~nm. Chains are modeled as freely jointed with no bending potential and \texttt{angle\_style none}, consistent with the high flexibility of intrinsically disordered regions.

\subsection{Simulation Protocol}

Simulations were performed using LAMMPS version 20240207 with NVE ensemble and Langevin thermostat. The baseline system contained 500 polymer chains of 24 beads each in architecture A4B20 within a cubic periodic box of 35 nm edge length, yielding a bead density of $\rho \approx 0.23$ $\sigma^{-3}$ consistent with physiological macromolecular crowding.

Each simulation proceeded through three phases. First, initial relaxation ramped $\epsilon$ from 0 to target value over 2 million timesteps to avoid configuration artifacts. Second, extended equilibration at constant $\epsilon$ and temperature ran for 5 million timesteps. Third, production ran 20 million timesteps with configurations recorded every 40,000 steps for analysis. Total simulation time per condition was 27 million timesteps corresponding to approximately 14 hours wall-clock time on c5.xlarge EC2 instances with MPI parallelization over 2 cores.

Chain length variants used adjusted box sizes: N=48 chains employed 100 nm cubic boxes, and N=96 chains employed 200 nm cubic boxes to accommodate longer chain spans.

\subsection{Computational Resources}

A total of 109 simulations were executed on Amazon Web Services EC2 c5.xlarge instances at 4 vCPUs and 8~GB RAM with Intel Xeon Platinum 8000-series. The campaign registered 135 simulations and 26 were cancelled after protocol revision. Sanity and backbone validation used 7 instances for 7 simulations. The A4B20 core $\epsilon \times T$ grid of 42 simulations used up to 21 instances. Valency and chain length variants of 24 simulations used up to 24 instances. The A8B16 fine $\epsilon$-$T$ grid of 14 simulations used 14 instances. The 1-1 vs.\ 1-2 interaction model comparison used up to 16 instances for 23 simulations where 16 ran and 7 were queued. A fleet of up to 32 simultaneous c5.xlarge instances at 128 vCPUs total enabled parallel execution across the parameter space. Each simulation required approximately 14 hours wall-clock time at 7M relaxation and 20M production steps on a single c5.xlarge instance using 2 MPI ranks.

\subsection{Parameter Space Exploration}

The complete parameter space included interaction strength $\epsilon$ from 3 to 15 $k_BT$ in increments of 1 $k_BT$ for the core grid with half-integer refinement at 6.5, 7.5, 8.5, 9.5, 10.5, 11.5 $k_BT$ near the phase boundary. Temperature ranged from 250 to 340 K in increments of 25 K for the core grid with 10 K refinement at 260, 270, 280, 290, 310 K near the boundary. Architectures were A2B22, A4B20, A6B18, A8B16, and A12B12. Chain lengths were N=24, 48, and 96 beads. Chain copy numbers ranged from 100 to 500. The core $\epsilon$-$T$ phase diagram for A4B20 was mapped with 42 simulations. Valency and chain length effects were explored across five architectures. The A8B16 phase boundary was refined with a fine grid in $\epsilon$ and $T$. A dedicated set compared 1-1 and 1-2 interaction models at matched conditions.

\subsection{Interaction Model Comparison}

The primary simulation campaign used the 1-2 sticker-spacer attraction model where the cross-type sticker-spacer pair interacts via an attractive soft cosine potential ramped to prefactor $A = -13.8$ while same-type pairs interact via repulsive Lennard-Jones. The 1-1 comparison set replaced this with the 1-1 sticker-sticker model where the same-type sticker pair carries the attractive soft potential while sticker-spacer interactions use repulsive Lennard-Jones. All other parameters including FENE bonds, $\epsilon$, $T$, chain geometry, box size, and integration scheme were held identical to enable controlled comparison.

\subsection{Order Parameters}

Phase separation was quantified using the fraction of chains in the largest cluster and the partition coefficient $K = \rho_{\text{dense}} / \rho_{\text{dilute}}$. Cluster assignment used inter-chain sticker-spacer contacts with a distance criterion of 1.0~nm, matching the cutoff of the attractive soft potential so that the order parameter reflects the same interaction range that drives phase separation. Radius of gyration $R_g$ and radial concentration profiles $\rho(r)$ provided additional characterization of condensate structure.

\begin{table}[H]
\centering
\caption{Simulation Parameters}
\label{tab:parameters}
\small
\begin{tabular}{@{}lp{0.52\textwidth}@{}}
\toprule
\textbf{Parameter} & \textbf{Value} \\
\midrule
Chain length baseline & N = 24 beads \\
Architecture baseline & A4B20 with 4 stickers and 20 spacers \\
Chain copies baseline & 500 chains \\
Box size baseline & 35 nm cubic periodic \\
Bead density & $\rho \approx 0.28$ $\sigma^{-3}$ \\
Interaction strength range & $\epsilon$ = 3--15 $k_BT$ integer plus fine grid at 0.5 $k_BT$ \\
Temperature range & $T$ = 250--340 K at 25 K plus fine grid at 10 K \\
Timestep & $\Delta t = 0.000125$ ns \\
Relaxation steps & 2M ramp plus 5M equilibration equals 7M steps \\
Production steps & 20M steps \\
Trajectory frames & 500 snapshots at every 40,000 steps \\
Valency variants & A2B22, A6B18, A8B16, A12B12 \\
Chain length variants & N = 48 in box 100 nm and N = 96 in box 200 nm \\
Interaction models & 1-2 sticker-spacer for main campaign and 1-1 sticker-sticker for comparison set \\
Total simulations & 109 active from 7 sanity plus 42 core plus 24 variants plus 14 fine plus 23 comparison \\
\bottomrule
\end{tabular}
\end{table}

\subsection*{Acknowledgments, Author Contributions, Declaration of Interests, and Funding}

\textbf{Acknowledgments.} The author would like to acknowledge Professor Carlos Castañeda for teaching the module that provided the framework for running the simulations. The author also acknowledges the BioInspired Institute for Living and Material Systems at Syracuse University and the NSF Research Traineeship program Emergent Intelligence Research for Graduate Excellence in Biological and Bio-Inspired Systems (EmIRGE-Bio), which hosts this course (\url{https://bioinspired.syr.edu/emirge-bio/}). The author is grateful to his parents for their support.

\textbf{Author Contributions.} G.R. conceived the study, performed all simulations, analyzed data, and wrote the manuscript.

\textbf{Declaration of Interests.} The author declares no competing interests.

\textbf{Funding.} None. This work received no grant or institutional funding.

\clearpage
\bibliographystyle{unsrtnat}
\bibliography{references}

\end{document}
